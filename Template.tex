
%این تمپلیت، نخستین بار برای بخش نوشتار نوزدهمین همایش علمی زرتشتیان سراسر کشور و توسط سپهر میزانیان تهیه و تنظیم شده است.

\documentclass[twoside]{article}
% Imported Packages ==================
\usepackage{graphicx}
\usepackage{amsfonts} 
\usepackage{amsmath}
\usepackage{indentfirst}
\usepackage{multirow}
\usepackage{fontspec}
\usepackage{fontawesome}
\usepackage[dvipsnames]{xcolor}
\usepackage[top=4cm,right=3cm,headheight=2cm]{geometry}
\usepackage{fancyhdr}
\usepackage{hyperref}
\usepackage{xepersian} 
% Font Settings ======================
\settextfont{B Nazanin}
\setlatintextfont{Times New Roman}
% Graphic Settings ====================
\graphicspath{ {Images/} }
\DeclareGraphicsExtensions{.pdf,.png,.jpg}
% Hyperref Settings ====================
\hypersetup{colorlinks=true}
\hypersetup{allcolors=red}
% Header and Footer Settings ==============
\fancyhf{}
\fancyhead[R]{نوزدهمین همایش علمی زرتشتیان سراسر کشور\\\textbf{\عنوان} - \نام\\\noindent\makebox[\linewidth]{\rule{\paperwidth}{1pt}}}
% اگر به صورت گروهی مقاله را نوشته اید، خط بالا را کامنت کرده و خط پایین را از حالت کامنت خارج کنید.
%\fancyhead[R]{نوزدهمین همایش علمی زرتشتیان سراسر کشور\\\textbf{\عنوان} -  \نام  و \همکار \\\noindent\makebox[\linewidth]{\rule{\paperwidth}{1pt}}}
\fancyfoot[C]{\thepage}
\renewcommand\headrulewidth{0pt}
\pagestyle{fancy}
% Command Definements ================
\renewcommand\refname{کتابنامه}
% فیلد های زیر را پر کرده، سپس کد را با کامپایلر XeLaTeX، کامپایل کنید. خواهید دید که هدر و قسمت آغازین مقاله، خود به خود کامل خواهند شد.
\newcommand{\عنوان}{عنوان ارائه}
\newcommand{\نام}{نام و نام خانوادگی}
\newcommand{\رشته}{رشته}
\newcommand{\دانشگاه}{دانشگاه}
\newcommand{\email}{example@gmail.com}
% تنها در صورتی که به صورت گروهی مقاله را نوشته اید، نیاز است تا دستور زیر را تعریف کنید. در غیر این صورت، اجازه بدهید به همین صورت باقی بماند.
\newcommand{\همکار}{نام و نام خانوادگی همکار}
%begin document =====================
\begin{document}
\begin{center}
\subsection*{به نام خداوند جان و خرد}
\vspace{5mm}
\title*{\textbf{\fontsize{50pt}{36pt}\selectfont \textcolor{BrickRed}{\عنوان}}}

\vspace{5mm}
{\fontsize{20pt}{36pt}\selectfont \نام}

\vspace{2mm}
% اگر به صورت گروهی مقاله را نوشته اید، تنها نیاز است تا رشته و دانشگاه یکی از افراد (نویسنده اصلی) را در جایگاه زیر وارد کنید. 
{\fontsize{15pt}{36pt}\selectfont دانشجوی رشته \رشته - دانشگاه \دانشگاه}
% اگر به صورت گروهی مقاله را نوشته اید، دو خط پایین را از حالت کامنت خارج کنید.

%\vspace{2mm}
%{\fontsize{15pt}{36pt}\selectfont با همکاری \همکار}

\vspace{2mm}
% اگر به صورت گروهی مقاله را نوشته اید، تنها نیاز است تا ایمیل یکی از افراد (نویسنده اصلی) را در جایگاه زیر وارد کنید. 
\lr{\email} \faEnvelope

\vspace{2mm}

\end{center}
\section*{چکیده} در این مکان، چکیده ارائه خود را بنویسید.

\vspace{5mm}
\noindent\textbf{واژگان کلیدی:} در این مکان، واژگان کلیدی ارائه خود را بنویسید.
\newpage
\section{مقدمه} در این مکان، مقدمه ارائه خود را بنویسید.
\section{بخش های بعدی} به همین ترتیب، نگارش مقاله خود را ادامه دهید.
\subsection{زیربخش ها}
زیربخش های بعدی را به همین شیوه وارد کنید.
\section{آموزش نحوه عکس گذاشتن، ارجاع به عکس و متن و نحوه فرمول نویسی}
\subsection{توضیحات کلی تمپلیت}
ابتدا یک دور فایل \lr{pdf} تمپلیت را بخوانید تا توضیحات کلی را متوجه شوید و قالب کلی آن را ببینید. سپس، مطابق با دستور العمل نوشته شده در متن تمپلیت و کامنت های موجود در کد لاتکس، اقدام به ویرایش آن و نگارش مقاله خود کنید. استفاده از \lr{\LaTeX} در نمرات مرحله پایانی همایش، اثر مثبت خواهد داشت.

درمورد نحوه نوشتن منابع در بخش کتابشناسی، چگونگی نوشتن پانویسهای انگلیسی، نحوه کشیدن جدول و ارجاع در متن به آن، سخت گیری نخواهد بود. به طور کلی، درصورتی که قصد دارید تا با استفاده از \lr{\LaTeX} مقاله خود را بنویسید، بجز در مواردی که پیشتر اشاره شد و همینطور مکان قرارگیری تصاویر، مجاز هستید استایل دلخواه خود را به کار گیرید. لطفا هدر را تغییر ندهید، درمورد نام ارائه و نویسندگان مطابق با دستور العمل تمپلیت پیش بروید و ارجاعات درون متنی به عکسها، جداول و منابع مانند مواردی باشد که پیشتر گفته شده و در ادامه گفته خواهد شد. توجه کنید پاراگرافهایی که تنها از یک خط تشکیل شده اند، \underline{نباید} فرورفتگی ابتدای پاراگراف را داشته باشند.
\subsection{نحوه اضافه کردن عکس به مقاله و نحوه ارجاع به آنها}
%نحوه اضافه کردن عکس در مقاله  ==============================================
عکس ها همگی باید در فولدی به نام \lr{Images} قرار داشته باشند. درصورتی که چنین نیست، تنظیمات مربوط به \lr{graphicspath} در ابتدای تمپلیت را تغییر دهید. توجه داشته باشید که در این تمپلیت، مکان عکسها به صورت استاندارد دیفالت خود لاتکس قرار میگیرند (تصاویر همیشه در بالای صفحات هستند.) در صورتی که احساس کنید این کار موجب ایجاد صفحه پرت در مقاله شما میشود، مجاز هستید از این استاندارد پیروی نکنید.

\begin{figure}
\centering
\includegraphics[width=7.62cm,height=7.62cm]{Fig1}
\caption{توضیحات تصویر \label{fig1}}
\end{figure}
\begin{figure}
\centering
\begin{tabular}{cc}
\includegraphics[width=5cm,height=5cm]{Fig2}&
\includegraphics[width=5cm,height=5cm]{Fig3}\\\\
(الف) & (ب) \\
\end{tabular}
\caption{توضیحات الف- شکل سمت راست، ب- شکل سمت چپ  \label{fig2}}
\end{figure}
نحوه ارجاع به شکل \ref{fig1} و شکل \ref{fig2} در متن مقاله
%پایان نحوه اضافه کردن عکس در مقاله  ==============================================
\subsection{توضیح چگونگی ارجاع به متون فارسی}
\noindent در ادامه، متنی نمونه از یک مقاله با موضوع تاریخ معاصر زرتشتیان به زبان فارسی آورده خواهد شد تا نحوه ارجاع به متون فارسی، توضیح داده شود.:

ایجاد انجمنهـا و مـدارس زمینـه پیشرفت اجتماعی و فرهنگی آنها را فـراهم نمـود. لغـو جزیه، که باری مالی بر دوش آنها بـود، باعـث آرامـش فکری و در نتیجه، توجه آنان به امور اقتـصادی گردیـد که این هـم ایجـاد تجارتخانـه هـا را بـه دنبـال داشـت.  زرتـشتیان تجارتخانـه هـای مهمـی چـون جمـشیدیان، جهانیان و یگـانگی را پایـه گـذاری نمودنـد و در امـور مختلفـی از جملـه زمینـداری و بانکـداری بـه فعالیـت پرداختند. درستکاری و نیک کرداری که زرتشتیان بـدان معروف بودند، عامل مهمی بود تا موقعیت آنها در بـین مردم و هیأت حاکمه بالا رود تا آنجا کـه مـورد اعتمـاد مردم و حکومت قـرار گرفتنـد. اربـاب جمـشید، بـانی تجارتخانه جهانیان، با اسـتفاده از پـشتوانه مـالی خـود توانست به عنوان نخستین نماینده زرتشتیان بـه مجلـس شورای ملی راه یابد و بـدین ترتیـب، پـس از گذشـت چندین قرن جامعه زرتشتی ایران از انـزوا بیـرون آمـد. \cite{Oshidari}

\subsection{توضیح چگونگی ارجاع به متون انگلیسی}
\noindent در ادامه، متنی نمونه از یک کتاب با موضوع تاریخ ساسانیان به زبان انگلیسی آورده خواهد شد تا نحوه ارجاع به متون انگلیسی توضیح داده شود.

مسکوکات ساسانی طلا و نقره و مس بودند اما نسبت بین ارزش آن‌ها در تمام دوره ساسانیان ثابت نبود. درهم نقره معمولاً به بین سه و شصت و پنج تا سه و نود و چهار گرم وزن داشت. این وزن بر اساس سکه‌های اشکانی و سکه‌های فنیقی است. در سکه‌های ساسانی نقش شاه همواره در یک طرف سکه حک شده‌است و در طرف دیگر معمولاً یک آتشدان است. \cite{Daryaee}
%نحوه اضافه کردن نقل قول در مقاله  ==============================================
\subsection{آموزش نقل قول در مقاله}
\noindent در ادامه، یک نقل قول آورده میشود تا نحوه آوردن نقل قول توضیح داده شود.
\begin{quote}\textbf{
من بسیار خوشحال هستم که در همایش علمی شرکت میکنم.\footnote{نحوه نوشتن پانویس در مقاله}
}\end{quote}
%پایان نحوه اضافه کردن نقل قول در مقاله  ==============================================
%نحوه اضافه کردن معادلات ریاضی در مقاله  ==============================================
\subsection{آموزش ریاضی نویسی و ارجاع به آنها در متن}
\noindent در ادامه، یک فرمول ریاضی آورده میشود تا نحوه نگارش فرمولهای ریاضی در متن و ارجاع به آنها مشخص شود.
  \begin{equation}
  \begin{aligned}
\label{eq1}
a_n &= 12 + 7 \int_0^n\Big(-\frac{1}{4}\big(e^{-4t_1} + e^{4t_1 - 8}\big)\Big) \,dt_1 \\
      &= 12 - \frac{7}{4} \int_0^n\big(e^{-4t_1} + e^{4t_1 - 8}\big) \,dt_1
  \end{aligned}
  \end{equation}
  \begin{equation*}
A_{m,n} = \begin{pmatrix}  a_{1,1} & a_{1,2} & \ldots & a_{1,n} \\ a_{2,1} & a_{2,2} & \ldots & a_{2,n} \\  {\normalsize \vdots} &  {\normalsize \vdots} & {\normalsize \ddots} &  {\normalsize \vdots} \\ a_{m,1} & a_{m,2} & \ldots & a_{m,n} \\  \end{pmatrix}
  \end{equation*}
همانطور که میبینید، معادله \ref{eq1} دارای شماره است اما ماتریس بالای این متن، خیر. زیرا در مقالات استاندارد، تنها معادلاتی دارای شماره گذاری هستند که در متن به آنها ارجاع وجود داشته باشد.
%پایان نحوه اضافه کردن معادلات ریاضی در مقاله  ==============================================
% نحوه نوشتن کتابشناسی =========================================================
% کتابشناسی حتما باید در صفحه جدیدی باشد.
\newpage
\begin{thebibliography}{100}
\bibitem{Oshidari} اوشیدری، جهانگیر (1355) \textbf{تاریخ پهلوی و زرتشتیان}، انتشارات ماهنامه هوخت
\begin{latin}
\bibitem{Daryaee} Daryaee, Touraj (2009), \textbf{Sasanian Persia: The Rise and Fall of an Empire} , I.B.Tauris

\end{latin} 

\end{thebibliography}

\end{document}
