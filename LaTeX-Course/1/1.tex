% این جلسه اول کارگاه لاتک کانون دانشجویان زرتشتی است.


\documentclass{report}

\usepackage{indentfirst}
\usepackage{fontawesome}
\usepackage[top=4cm]{geometry}
\usepackage{graphicx}
\usepackage{hyperref}
\graphicspath{{Images/}}

\hypersetup{colorlinks=true}
\hypersetup{allcolors=blue}



\begin{document}
Hello World!



\tableofcontents

\listoffigures



\part{First}

\chapter{First Chapter}

\section{First Section}

\subsection*{First Subsection \faAmazon}

Zoroastrianism currently has some 125,000 adherents worldwide with the majority 
living in India, mostly in Mumbai and Gujarat \autoref{fig1} (estimated at 60,000 for the as yet 
unreleased census of 2011). In South Asia the Zoroastrians are known as the “Parsis” 
(see Hinnells, “The Parsis,” this volume).

\noindent Since World War II their numbers have been 
in rapid decline (there were just under 115,000 Parsis in pre‐Partition India in 1941) 
and the Indian media report dire predictions according to which this trend will continue 
in the upcoming decades. The second largest group of Zoroastrians is to be found in 
Iran, from where the Parsis relocated in the aftermath of the Arab invasions in the 
mid‐7th century ce and the Islamization of the country in the following centuries 
\newline
(see Daryaee, “Zoroastrianism under Islamic Rule,” this volume). 

\newpage

Fewer than 20,000 
Zoroastrians currently reside in Iran, where they are recognized as a religious minority 
by the constitution.\footnote{\hypersetup{allcolors=green} See \autoref{sss1}}

\begin{figure}[h]
\includegraphics[width=0.5\linewidth]{Figure1.jpg}
\caption{fskjgbdkjassksajk}
\label{fig1}
\end{figure}
\begin{figure}[h]
\includegraphics[width=0.3\linewidth]{New Folder/Picture2.png}
\caption{fskjgbdkjassksajk}
\end{figure}

\chapter{Second} \label{sss1}

\subsubsection{First Subsubsection}
(see Stausberg, “Zoroastrians in Modern Iran,” this volume)

\end{document}
