
% طرح درس دوره لاتکس کانون دانشجویان زرتشتی

\documentclass{article}
% Imported Packages ==================
\usepackage{graphicx}
\usepackage{amsfonts} 
\usepackage{amsmath}
\usepackage{indentfirst}
\usepackage{multirow}
\usepackage{fontspec}
\usepackage{fontawesome}
\usepackage[dvipsnames]{xcolor}
\usepackage[top=4cm,right=3cm]{geometry}
\usepackage{fancyhdr}
\usepackage{hyperref}
\usepackage{xepersian} 
% Font Settings ======================
\settextfont[Scale=1.2]{B Nazanin}
\setlatintextfont{Times New Roman}
% Graphic Settings ====================
\graphicspath{ {Images/} }
\DeclareGraphicsExtensions{.pdf,.png,.jpg}
% Hyperref Settings ====================
\hypersetup{colorlinks=true}
\hypersetup{allcolors=red}
%begin document =====================
\begin{document}
\textbf{به نام خداوند جان و خرد}
\begin{flushright}
\vspace{5mm}
\title*{\textbf{\fontsize{31pt}{36pt}\selectfont \textcolor{BrickRed}{طرح درس دوره آموزش}}}
\end{flushright}
\title*{\textbf{\fontsize{180pt}{36pt}\selectfont \textcolor{BrickRed}{\lr{\LaTeX}}}}
\begin{flushleft}
\title*{\textbf{\fontsize{31pt}{36pt}\selectfont \textcolor{BrickRed}{کانون دانشجویان زرتشتی}}}
\end{flushleft}
\vspace{5mm}

لاتکس یا لاتک، یک زبان استایل‌دهی برای حروفچینی کامپیوتری و یکی از مهارت‌های پایه‌ای یک دانشجو است، زیراکه در مجلات معتبر علمی و همینطور در پایان نامه‌های دانشگاهی، معمولا از ابزارهایی چون وورد استفاده نمی‌شود، چرا که رعایت تمام استاندارد‌های مقاله‌نویسی، فونت‌‌‌‌‌‌‌‌‌‌‌‌‌‌‌‌‌‌‌‌‌ها، مکان تصویر، چگونگی تهیه کتابشناسی و... در آن ابزارها تقریبا ناممکن است. ابزار لاتکس شبیه به یک زبان برنامه نویسی است. ابزار قدرتمندی است که نه تنها نوشتن فرمول‌های ریاضی، تهیه فهرست و واژه‌نامه را به آسانی ممکن می‌سازد، بلکه به ما این امکان را می‌دهد تا با نوشتن تنها چند خط کد ساده، به طور خودکار استانداردهای نگارشی مجله‌ای که در حال نگارش مقاله برای آن هستیم یا پایان‌نامه دانشگاهی که در آن تحصیل می‌کنیم، رعایت شود و ما تمام تمرکز خود را روی متن بگذاریم. در واقع این ابزار، به ما تضمین می‌دهد که استایل مقاله یا پایان‌نامه ما حتما استاندارد باشد و ما را دیگر درگیر رعایت استانداردهای گوناگون نمی‌کند. لاتکس، به ما اجازه می‌هد تا تنها با \underline{یک خط کد ساده}، فهرست مطالب و فهرست مراجع را متناسب با استانداردهای مقاله نویسی تهیه کنیم. همینطور، لاتکس به دلیل اینکه از کد تشکیل شده است، بسیاری از مشکلات وورد نظیر جابجاشدن متن هنگام انتقال از دستگاهی به دستگاه دیگر را ندارد. همینطور، قابلیت استفاده از سامانه‌های کنترل نسخه\LTRfootnote{Version Control Systems} که مخصوص پروژه‌های بزرگ برنامه نویسی است، برای لاتکس نیز پشتیبانی می‌شود که در نگارش متون علمی به صورت گروهی و یا به‌روزرسانی کتاب یا پایان‌نامه، بسیار کمک کننده است. چنین قابلیتی مطلقا برای وورد یا ابزارهای مشابه وجود ندارد. در این دوره، ما تلاش می‌کنیم تا با آموزش مهارت‌های پایه‌ای لاتکس، دانشجویان را با این ابزار قدرتمند آشنا کنیم.
\newpage
دوره آموزشی لاتکس کانون دانشجویان زرتشتی، در دو جلسه 100 دقیقه‌ای و در روزهای 19 و 22 بهمن‌ماه 1401 خورشیدی برگزار می‌شود که موارد آموزشی آن به شرح زیر است:




\section{معرفی لاتکس، نگارش متون، گذاشتن عکس و مقدمات نوشتن فرمول‌های ریاضی}
\subsection*{معرفی \lr{\LaTeX}: 15 دقیقه}
\begin{enumerate}
\item چرا باید لاتکس را بیاموزیم؟
\item دغدغه کسی که متون علمی می‌نویسد، استایل و ظاهر است یا محتوا
\end{enumerate}
\subsection*{معرفی ادیتورهای لاتکس: 10 دقیقه}
\begin{enumerate}
\item \lr{TeX Live}، بهترین ادیتور
\item آشنایی با ادیتور \lr{TeXstudio}
\item آشنایی با ادیتور \lr{TeXworks editor}
\item ادیتور آنلاین \lr{Overleaf}
\end{enumerate}
\subsection*{نخستین کدها: 10 دقیقه}
\begin{enumerate}
\item سلام دنیا!
\item آموزش \lr{$\backslash$section}، \lr{$\backslash$chapter}، \lr{$\backslash$subsection} و \lr{$\backslash$part}
\item آموزش کامنت گذاشتن در کد
\item نوشتن متن نمونه در لاتکس
\item پانویس در لاتکس
\item ایجاد صفحه جدید در لاتکس
\end{enumerate}
\subsection*{ارجاع دادن به خود متن در لاتکس: 10 دقیقه}
\begin{enumerate}
\item آموزش \lr{$\backslash$label}
\item آموزش \lr{$\backslash$ref} و \lr{$\backslash$autoref}
\end{enumerate}
\subsection*{مفهوم پکیج و استفاده از چند پکیج نمونه: 20 دقیقه}
\begin{enumerate}
\item پکیج چیست؟
\item نمونه 1: رفع فرورفتگی ابتدای پاراگراف با پکیج \lr{indentfirst}
\item نمونه 2: تعیین ابعاد صفحه با پکیج \lr{geometry}
\item نمونه 3: آیکون‌های اینترنتی با پکیج \lr{fontawesome}
\end{enumerate}
\subsection*{گذاشتن تصویر در لاتکس: 25 دقیقه}
\begin{enumerate}
\item پکیج \lr{graphicx}
\item گذاشتن تصویر در لاتکس
\item تغییر ابعاد تصویر
\item توضیح مشکل مکان قرارگیری تصاویر 
\item نوشتن کپشن برای عکس‌ها
\item ارجاع به عکس در متن با کمک \lr{$\backslash$ref}
\end{enumerate}
\subsection*{معرفی پکیج \lr{hyperref}: 10 دقیقه}
\begin{enumerate}
\item فهرست مطالب در لاتکس
\item قهرست تصاویر در لاتکس
\item استفاده از پکیج \lr{hyperref}
\item تغییر تنظیمات ابتدایی پکیج \lr{hyperref} 
\end{enumerate}

\section{مقدمات نوشتن فرمول‌های ریاضی، فهرست مراجع و فارسی‌نویسی در لاتکس}
\subsection*{استایل‌دهی به متون در لاتکس: 10 دقیقه}
\begin{enumerate}
\item بولد کردن متن
\item ایتالیک کردن متن
\item خط کشیدن زیر متن
\item تغییر رنگ متن
\end{enumerate}
\subsection*{نوشتن فرمولهای ریاضی در لاتکس: 25 دقیقه}
\begin{enumerate}
\item نوشتن معادله در متن
\item معرفی محیط \lr{equation}
\item چگونگی نوشتن کسر، ریشه و توان
\item چگونگی نوشتن حروف یونانی
\item معادلات چندخطی و معرفی محیط \lr{align}
\item ارجاع به معادله در متن
\end{enumerate}
\subsection*{آیتم در لاتکس: 10 دقیقه}
\begin{enumerate}
\item آیتم‌های غیرعددی و معرفی محیط \lr{itemize}
\item آیتم‌های عددی و معرفی محیط \lr{enumerate}
\end{enumerate}
\subsection*{تهیه فهرست مراجع در لاتکس: 25 دقیقه}
\begin{enumerate}
\item معرفی محیط \lr{thebibliography} و مرجع نویسی در خود فایل اصلی
\item معرفی \lr{$\backslash$cite} و آرگومان‌های اختیاری آن
\item معرفی \lr{BibTeX} و فایلهای با پسوند \lr{.bib}
\item استایل‌های مختلف مراجع
\end{enumerate}
\subsection*{فارسی‌نویسی در لاتکس: 20 دقیقه}
\begin{enumerate}
\item معرفی پکیج \lr{xepersian}
\item ست کردن فونت‌ فارسی
\item چگونه در متن فارسی، انگلیسی بنویسیم؟
\item پانویس‌های فارسی و انگلیسی
\item ارجاع دهی به منابع فارسی در لاتکس به هر دو شیوه
\end{enumerate}
\subsection*{چگونگی استفاده از پمپلیت مقالات همایش علمی: 10 دقیقه}
\begin{enumerate}
\item  توضیح تمپلیت مقالات همایش علمی
\item نمونه نوشتن مقاله در تمپلیت
\end{enumerate}



توجه داشته باشید که در انتهای هر جلسه، کدهای آموزشی با شرکت‌کنندگان در کلاس به اشتراک گذاشته خواهد شد. همچنین، در انتهای هر جلسه، یک تمرین داده میشود که تا جلسه آینده باید کدهای مربوط به آن تمرین را تحویل دهید. در نهایت، درصورتی که نمره شما در تمرینات کلاس از 70 درصد بیشتر باشد، از جانب کانون دانشجویان زرتشتی \underline{گواهی شرکت در دوره} دریافت خواهید کرد.

\vspace{5cm}
\begin{center}
به امید فردایی بهتر

کانون دانشجویان زرتشتی
\end{center}
\end{document}
