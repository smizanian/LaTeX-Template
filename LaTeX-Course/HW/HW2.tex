\documentclass{report}
% Imported Packages ==================
\usepackage{graphicx}
\usepackage{amsfonts} 
\usepackage{amsmath}
\usepackage{indentfirst}
\usepackage{multirow}
\usepackage{fontspec}
\usepackage{fontawesome}
\usepackage[dvipsnames]{xcolor}
\usepackage[top=4cm,right=3cm,headheight=2cm]{geometry}
\usepackage{fancyhdr}
\usepackage{hyperref}
\usepackage{xepersian} 
% Font Settings ======================
\settextfont{B Nazanin}
\setlatintextfont{Times New Roman}
\renewcommand{\labelitemi}{$\bullet$}
% Graphic Settings ====================
\graphicspath{ {Images/} }
\DeclareGraphicsExtensions{.pdf,.png,.jpg}
% Hyperref Settings ====================
\hypersetup{colorlinks=true}
\hypersetup{allcolors=red}
%begin document =====================
\begin{document}
\part*{\lr{KDZ LaTeX Course - HW2}}

\tableofcontents

\chapter{مقدمات}
\section{جسم سیاه}
یک جسم سیاه \LTRfootnote{Black body} ایده آل تمام پرتوهایی را که در طول موج ها و زوایای مختلف دریافت کند جذب می کند. برای
مثال یک محفظه در نظر بگیرید که با دیواره از محیط بیرون جدا شده و تنها از طریق تابشی که از یک روزنه
کوچک وارد می شود با محیط اطراف انرژی مبادله می کند(کاواک). این کاواک را در تعادل ترمودینامیکی در
یک دمای معلوم در نظر بگیرید. هر تابشی که از طریق یک روزنه کوچک در دیواره کاواک وارد شود توسط
دیواره ها جذب می شود و بازتاب می کند. تابشی که از این روزنه بیرون آید معیاری از تابش درون کاواک
است، مستقل از خواص روزنه یا ترکیب شیمیایی دیواره. طبق قانون کیرشهوف در یک جسم سیاه در حال
تعادل، ضرایب جذب و تابش برابرند و این در تمام طول موج ها صادق است. \cite{birney2006observational}

یک جسم سیاه در نزدیکی صفر کلوین واقعا سیاه است چون تابش ناچیزی دارد اما در دماهای بالتر یک تابش
با توزیع ویژه دارد. بسیاری از چشمه های نجومی با تقریب خوبی جسم سیاه هستند مثل تابش زمینه کیهانی.
پیش از آن که به توزیع تابش که با قانون پلنک بیان می شود بپردازیم لزم است درباره تقریب تعادل
ترمودینامیکی بحث کنیم. برخورد بین ذرات در یک آنسامب آن را به سمت تعادل ترمودینامیکی سوق می دهد.
در نتیجه در حالتی که چگالی عددی بزرگ باشد، آهنگ برخورد بسیار بالست و اغلب به تعادل نزدیک هستیم.
به طور کلی اگر نسبت انرژی فوتون در یک محیط به انرژی متوسط ذره خیلی کوچکتر از یک باشد، در تعادل
ترمودینامیکی موضعی
 هستیم. تعادل ترمودینامیکی موضعی حالتی است که در آن نسبت ضریب جذب و
تابش محیط، جمعیت ترازهای اتمی و توزیع سرعت ذرات هر سه را می توان با یک دما توصیف کرد. ممکن
است در یک محیط مثل الکترونها در تعادل ترمودینامیکی موضعی \LTRfootnote{\textit{ Local Thermodynamic Equilibrium (LTE)}} باشند اما یونها در تعادل نباشند. بطور کلی
 اگر طول همدما شدن
 (\textbf{\underline{مسافتی که ذره داغ باید طی کند تا هم دما شود}}) در یک محیط نسبت به پویش آزاد
میانگین بزرگ باشد از شرایط تعادلی دور هستیم. نمونه ای از این گونه محیط ها جو بیرونی ستاره ها، تاج
خورشید، سحابی های سیاره نما، محیط های میان ستاره ای و میان کهکشانی است. در این شرایط متوسط
انرژی فوتون قابل مقایسه یا بزرگتر از انرژی ذره است. یک ذره در شرایط نا تعادلی مسافت طولنی طی
می کند تا یک برخورد انجام دهد و به محیطی می رسد که متوسط انرژی ذرات آن نسبت به محیط برخورد
قبلی متفاوت است. در چنین شرایطی تعادل موضعی نیست. \cite{leblanc2011introduction, chandrasekhar1957introduction, carroll2007introduction}

\section{قانون پلانک}
توزیع تابش یک جسم سیاه به تنهایی از ترمودینامیک بدست نمی آید. پلنک با فرض گسسته بودن سطوح
انرژی موفق شد رابطه توزیع تابش را که به قانون پلنک معروف است اثبات کند. این قانون در حد انرژی کم
یا زیاد به قانون ریلی-جینز یا قانون وین تبدیل می شود. شدت تابش بر حسب طول موج یا بسامد برای یک
جسم سیاه در دمای \lr{T} برابر است با: \cite[صفحه 5]{leblanc2011introduction}

\begin{equation} \label{planck}
B_\lambda d_\lambda = \frac{2hc^2}{\lambda^5}\times\frac{1}{e^\frac{hc}{\lambda K_B T} - 1}
\end{equation}

\noindent در \autoref{planck}، \lr{h} برابر با ثابت پلانک، \lr{T} برابر با دمای جسم، $K_B$ برابر با ثابت بولتزمن، \lr{c} برابر با سرعت نور و $\lambda$ برابر با طول موج است.

\section{طرح پراش تلسکوپ}

در اثر پراش نور در دهانه تلسکوپ، مقداری از نور چشمه نقطه ای در طرحی که به ویژگی های دهانه بستگی
دارد پخش می شوند (منظور از ویژگی ها شکل، ضریب عبور و اختلف فاز بر حسب مکان است). در حد
تلسکوپ های نجومی، با پراش در تقریب فرانهوفر سر و کار داریم یعنی فرض می کنیم که طرح پراش
(آشکارساز) در فاصله دوری از دهانه تلسکوپ قرار دارد. پراش ویژگی ذاتی اپتیک است و در تلسکوپ های
فضایی نیز وجود دارد. ابتدا روی یک دهانه دایره ای انتگرال گیری می کنیم تا 
میدان الکتریکی روی پرده بدست آید.

\begin{align} \label{int}
U(P) & = C\int^a_0 \int^{2\pi}_0 e^{-i\kappa \rho \omega \cos(\theta - \psi)} \rho d\rho d\theta \\
& = 2\pi C \int^a_0 J_0(\kappa \rho \omega)\rho d\rho
\end{align}

در اینجا  $\rho$  و $\theta$ مختصات روی دهانه تلسکوپ هستند. چون دهانه اغلب به شکل همگن تابش نور را دریافت
می کند، آن را با یک ثابت از انتگرال بیرون بردیم. پارامتر \lr{U} بردار میدان الکتریکی روی پرده (آشکارساز) را
نشان می دهد. پارامتر $\omega^2 = p^2 + q^2$ است که \lr{p} و \lr{q} مختصات روی پرده هستند. در رابطه \ref{int}، $J_0$ تابع بسل مرتبه 0 است.

\subsection{عوامل اثرگذار روی تغییر طرح پراش}

\begin{enumerate}
\item دید نجومی
\item ابیراهی های اپتیکی
\end{enumerate}

\subsubsection{فهرست ابیراهی های اپتیکی مشهور}
\begin{itemize}
\item ابیراهی رنگی
\item ابیراهی کروی
\item ابیراهی کما
\item آستیگماتیزم
\end{itemize}

\newpage

\begin{latin}
\bibliographystyle{ieeetr}
\bibliography{bib.bib}	
\end{latin}
\end{document}