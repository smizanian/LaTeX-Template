\documentclass{report}

\usepackage{xepersian}

\settextfont{B Nazanin}
\setlatintextfont{Times New Roman}
\renewcommand{\labelitemi}{$\circ$}

\begin{document}
در این بخش مهمترین ابیراهی های یک سامانه نوری مثل تلسکوپ را بررسی می کنیم. ابیراهی های نوری \cite{art1}
ناشی از کیفیت بد عدسی یا آینه نیستند: آنها نتیجه قوانین پایه فیزیک نور هستند و هر قدر هم که قطعات

\lr{Hello World!}

اپتیکی با کیفیت ایده آل ساخته شوند این مشکلت که ناشی از طراحی تلسکوپ هستند وجود دارند. به عبارت
دیگر تمام تلسکوپ ها در روی زمین یا در فضا مقداری ابیراهی نوری دارند. پس ابیراهی های نوری ویژگی
های سامانه اپتیکی هستند و \textbf{\underline{خطا محسوب نمی شوند.}} برای نمونه ضریب شکست شیشه تابع طول موج
است: ضریب شکست محیط برای نور در طول موج های کوچک تر بزرگ تر است، پس نور آبی و قرمز در \cite{daryaee2014sasanian}
نام دارد. سایر ابیراهی نقاط مختلفی کانونی می شوند. این یک نمونه ابیراهی نوری است که ابیراهی رنگی \LTRfootnote{gfhfsg}
ها هم به همین ترتیب دلیل فیزیکی دارند. پیداست که طراحی و ساخت یک سامانه نوری بدون هیچ ابیراهی از
آرزوهای بزرگ منجمان است اما حتی در صورت وجود چنین تلسکوپی در بیرون جو باز هم کنتراست تصویر
(تضاد نوار های تاریک و روشن) کمتر از تصویر اصلی است که علت آن پراش است. پس به خاطر داشته
باشیم که با چه محدودیت هایی برای ثبت یک تصویر ایده آل مواجه هستیم و با شناخت ابیراهی های نوری
هنگام تصمیم گیری برای انتخاب سامانه اپتیکی تلسکوپ انتظارات واقع گرایانه ای داشته باشیم

\begin{equation}
	e = \frac{hc}{\lambda}
\end{equation}

\begin{enumerate}
	\item \lr{a}
	\item \lr{b}
	\item \lr{c}
\end{enumerate}

\begin{itemize}
	\item \lr{a}
	\item \lr{b}
	\item \lr{c}
\end{itemize}

\begin{latin}
\bibliographystyle{ieeetr}
\bibliography{ref.bib}
\end{latin}



\end{document}